\documentclass[12pt,letterpaper]{article}
\usepackage[utf8]{inputenc}
\usepackage[spanish]{babel}
\usepackage{amsmath}
\usepackage{amsfonts}
\usepackage{amssymb}
\usepackage{natbib}
\usepackage{graphicx}
\usepackage{cite}
\usepackage{wrapfig}
\usepackage{afterpage}
\setcitestyle{square}
\usepackage[export]{adjustbox}
\usepackage[left=2cm,right=2cm,top=2cm,bottom=2cm]{geometry}
\usepackage{afterpage}
\usepackage{setspace}
\spanishdecimal{.}

\author{C. Iván Pineda S.\textsuperscript{1} }
\title{Práctica 6\\ Taller de Modelación Numérica}
\date {\textit{\textsuperscript{1}Universidad Nacional Autónoma de México}
\\ 17 de Abril de 2017}

\begin{document}
\maketitle

Ejercicio: Implementar funciones para los métodos de Runge-Kutta de 4$^{\circ}$
orden en 2 y 3 dimensiones para resolver :

a) El PVI:

\[
f(x,y) = \left\lbrace
\begin{array}{ll}
y''+xy'+y=0, & x\in[0,10] \\
y(0)=1  & y'(0)=2
\end{array}
\right.
\]

b)Atractor de Rösler:

\[
x'=-y-z
\]

\[
y'=x+Ay
\]

\[
z'=B+z(x-c)
\]

Donde:

\[
A=0.2 \qquad A=0.1
\]

\[
B=0.2 \qquad B=0.1
\]

\[
C=5.7 \qquad C<14
\]

c)Atractor de Lorenz:

\[
x'=\sigma (y-x) \qquad \sigma=10
\]

\[
y'=x ( \rho - z ) - y \qquad \rho=28
\]

\[
z'=-\beta z+xy \qquad \beta=8/3
\]

$\cdot$ Graficar cada una de las soluciones en el dominio correspondiente y el
retrato fase.

\end{document}
